% Conserta o indent para alinhar conforme padrão de normalização
\addtocontents{toc}{\cftsetindents{chapter}{1.5cm}{3em}}
\addtocontents{toc}{\cftsetindents{section}{0cm}{1.5cm}}
\addtocontents{toc}{\cftsetindents{subsection}{0cm}{1.5cm}}
\addtocontents{toc}{\cftsetindents{subsubsection}{0cm}{1.5cm}}
\addtocontents{toc}{\protect\renewcommand{\protect\cftchapteraftersnum}{\enskip\textemdash\enskip}}


\printbibliography[title={REFERÊNCIAS}, heading=bibintoc] %

% ---
% Inicia os apêndices
% Para as seções nos apêndices usar o comando \section* que não adiciona a seção no sumário.
% ---
\appendix

\chapterstyle{apendices}

% ----------------------------------------------------------
\chapter{Derivação do Vetor de Médias}
% ----------------------------------------------------------


Seja o \(VAR(1)\) bivariado representado em sua forma reduzida:

\begin{equation}
X_t = A_0 + A_1X_{t-1} + e_t, 
\end{equation}

ou em notação matricial: 

\begin{equation}
\label{eq:var_reduz_matrix}
\begin{aligned} \begin{bmatrix}
z_t \\ y_t \end{bmatrix} &=\begin{bmatrix}
a_{10} \\ a_{20}\end{bmatrix} +
\begin{bmatrix}a_{11} & a_{12} \\a_{21} & a_{22}
\end{bmatrix}\begin{bmatrix} z_{t-1} \\ y_{t-1}
\end{bmatrix}  +\begin{bmatrix} e_{1t} \\ e_{2t}
\end{bmatrix}\end{aligned}
\end{equation}

Se

\begin{equation}
\left\{ 
\begin{array}{l}

\begin{bmatrix}e_{1t} \\e_{2t}
\end{bmatrix}\equiv B^{-1}\varepsilon_t =\begin{bmatrix}
\frac{\varepsilon_{zt} - b_{12} \varepsilon_{yt}}{1 - b_{12}b_{21}} \\[5pt]\frac{\varepsilon_{yt} - b_{21} \varepsilon_{zt}}{1 - b_{12}b_{21}}\end{bmatrix}
\\
\varepsilon_{zt} \sim \mathcal{RB}(0, \sigma^2_z) 

\end{array} 
\right.
\end{equation}

Então:
\begin{equation} \mathbb{E}(e_t) = \textbf{0} \end{equation}

Se, sob estacionariedade, para uma variável qualquer:

\begin{equation}
 \mathbb{E} \left( x_t \right) = \mathbb{E} \left( x_{t-1} \right) = \mu, \, \forall \, t \in \mathbb{Z} = \{0, \pm1, \pm2, ...\}
\end{equation}

Então, similarmente para \(z_t\) e \(y_t\), tem-se que os valores das séries flutuam em torno de suas médias de modo que \(\mathbb{E} \left( y_t \right) = \mathbb{E} \left( y_{t-1} \right) = \bar{y}\) e \(\mathbb{E} \left( z_t \right) = \mathbb{E} \left( z_{t-1} \right) = \bar{z}\).
 
Aplicando-se o operador de expectância sobre ambos os termos da equação \eqref{eq:var_reduz_matrix}, tem-se: 

\begin{equation}
\label{eq:expect_var}
\mathbb{E} \begin{bmatrix}z_t \\y_t\end{bmatrix} 
= 
\mathbb{E}\begin{bmatrix}a_{10} \\ a_{20}\end{bmatrix} 
+
\mathbb{E}\begin{bmatrix}
a_{11} & a_{12}\\a_{21} & a_{22}
\end{bmatrix}
\begin{bmatrix}z_{t-1} \\ y_{t-1}\end{bmatrix}
+
\mathbb{E}\begin{bmatrix}e_{1t} \\ e_{2t}\end{bmatrix}
\end{equation}

Uma vez que: 

\begin{equation}
\left\{ 
\begin{array}{l}
\mathbb{E} \begin{bmatrix}z_t \\y_t\end{bmatrix} 
= 
\begin{bmatrix}\mathbb{E}(y_t) \\\mathbb{E}(z_t)
\end{bmatrix} 
=
\begin{bmatrix}\bar{y} \\\bar{z}\end{bmatrix}
\\[15pt]
\mathbb{E} \begin{bmatrix}a_{10} \\a_{20}\end{bmatrix}
=
\begin{bmatrix}a_{10} \\a_{20}\end{bmatrix}
\\[15pt]
\mathbb{E}\begin{bmatrix}
a_{11} & a_{12}\\a_{21} & a_{22}\end{bmatrix}
\begin{bmatrix}z_{t-1} \\ y_{t-1}\end{bmatrix}
=
\begin{bmatrix}
a_{11} & a_{12}\\a_{21} & a_{22}\end{bmatrix}
\mathbb{E}\begin{bmatrix}z_{t-1} \\ y_{t-1}\end{bmatrix}
\\[15pt]
=
\begin{bmatrix}
a_{11} & a_{12}\\a_{21} & a_{22}\end{bmatrix}
\begin{bmatrix} \mathbb{E}(z_{t-1})
\\ \mathbb{E}(y_{t-1})\end{bmatrix}
~~ = 
\begin{bmatrix}
a_{11} & a_{12}\\a_{21} & a_{22}\end{bmatrix}
\begin{bmatrix}\bar{y} \\\bar{z}\end{bmatrix}
\end{array} 
\right.
\end{equation}

Então a equação  \eqref{eq:expect_var} pode ser reescrita da seguinte maneira:

\begin{equation}
\begin{bmatrix}
\bar{z} \\\bar{y}
\end{bmatrix}=
\begin{bmatrix}a_{10} \\a_{20}\end{bmatrix}
+\begin{bmatrix}a_{11} & a_{12} \\ a_{21} & a_{22}
\end{bmatrix}\begin{bmatrix}\bar{z} \\\bar{y}
\end{bmatrix}\end{equation}

De onde os valores de longo prazo da série podem ser encontrados mediante resolução do sistema linear:

\begin{equation}\begin{bmatrix}
\bar{z} \\\bar{y}
\end{bmatrix}-\begin{bmatrix}
a_{11} & a_{12} \\a_{21} & a_{22}
\end{bmatrix}\begin{bmatrix}\bar{z} \\\bar{y}
\end{bmatrix}=\begin{bmatrix}a_{10} \\a_{20}
\end{bmatrix}\end{equation}

\begin{equation}\left(\begin{bmatrix}1 & 0 \\0 & 1
\end{bmatrix}-\begin{bmatrix}
a_{11} & a_{12} \\a_{21} & a_{22}
\end{bmatrix}\right)\begin{bmatrix}
\bar{z} \\\bar{y}\end{bmatrix}
=\begin{bmatrix}a_{10} \\a_{20}
\end{bmatrix}\end{equation}

\begin{equation}
\underbrace{
\begin{bmatrix}\bar{z} \\\bar{y}
\end{bmatrix}
}_{\equiv \bar{X}}
=
\left(
\underbrace{
\begin{bmatrix}1 & 0 \\0 & 1\end{bmatrix}
}_{\equiv I}
-
\underbrace{
\begin{bmatrix}a_{11} & a_{12} 
\\a_{21} & a_{22}\end{bmatrix}
}_{\equiv A_1}
\right)^{-1}
\underbrace{
\begin{bmatrix}a_{10} \\a_{20}\end{bmatrix}
}_{\equiv A_0}
\end{equation}

% ----------------------------------------------------------
% \chapter{Visualizações das pilhas de areia}\label{caleidoscopios}
% % ----------------------------------------------------------
% \lipsum[10]


% \section*{Visualização da característica fractal da pilha de areia}
% \lipsum[10]

% ---
% Inicia os anexos
% ---
\addtocontents{toc}{\protect\renewcommand\protect\cftappendixname{\nomeanexo~}}
\appendix
\chapterstyle{anexos}

% Imprime uma página indicando o início dos apêndices





% % ----------------------------------------------------------
% \chapter{Anexos}
% % ----------------------------------------------------------
% \lipsum[5]

% % ----------------------------------------------------------
% \chapter{outro anexos}

% \lipsum[5]

% \section*{seção no anexo}

% \lipsum[5]

% \section*{Outra seção no anexo}

% \lipsum[5]

