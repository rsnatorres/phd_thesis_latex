\chapterstyle{texto}


\chapter{Referencial Teórico} 
\label{referencial_teorico}

\section{Introdução}
\bigskip

(falar de juros e não de investimento)


\clearpage
\section{Incerteza}
\label{referencial_teorico/incerteza}

O conceito de incerteza é uma das inovações teóricas mais revolucionárias do paradigma pós-keynesiano. A consideração da incerteza revelará a precariedade do conhecimento sobre o qual ocorre o processo de tomada de decisão \parencite{carvalho_keynes_2020}. Há duas formas de tratamento da incerteza pela vertente pós-keynesiana: i) Ontológica; ii) Epistemológica. Essa seção visa expor essas duas formas tal como entendida por estes autores.

\subsection{Incerteza Ontológica}

Essa forma de incerteza dispõe o conceito em termos de não-ergodicidade, conforme exposto nos trabalhos de (DAVIDSON, 1998, 1991a). A terminologia advém da teoria moderna dos processos estocásticos (DEQUECH, 1999), onde ergodicidade é uma propriedade dos processos estocásticos que garante a sua estabilidade, permitindo auferir distribuições de probabilidade.

Argumentam os autores pós-keynesianos, que o ambiente onde operam as variáveis econômicas são ambientes não-ergódigos \parencite{carvalho_keynes_2020}. Tal assertiva equivale a dizer que não há estabilidade nos processos econômicos, seja porque as variáveis econômicas são não-estacionárias, ou seja, independentes do tempo (AMADO, 2020, p.50), seja porque a ocorrência de mudanças estruturais é frequente (DEQUECH, 1999). Ainda que fossem estáveis, certos processos econômicos, como as decisões de investimento, não são repetitivos o suficiente para permitir um processo de descoberta ou aprendizado por tentativa e erro\footnote{"Se os processos estocásticos forem estáveis o bastante, a observação repetida levará ao conhecimento de seus padrões subjacentes" \parencite[pg. 20]{carvalho_keynes_2020}}. Em um caso extremo, ainda que os processos econômicos fossem estáveis e ocorressem com alta frequência, as condições nas quais esses processos se desenvolvem podem mudar devido a existência de experimentos cruciais (ou decisões cruciais). Com terminologia emprestada de SHAKLE (???), \textcite{carvalho_keynes_2020} e AMADO (2020) referem-se aos experimentos cruciais como a realização de eventos (ou experimentos) que destroem as condições de sua existência, impossibilitando a derivação de relações de frequência. Não é como no jogo de roleta, onde as realizações são independentes e as condições iniciais são restauradas a cada lance. 

